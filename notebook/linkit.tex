\documentclass[10pt, a4paper]{article}
\usepackage{graphicx}
\usepackage{url}
\usepackage{hyperref}
\usepackage{xcolor}
\usepackage{framed}
\usepackage{listings}

\colorlet{shadecolor}{blue!20}

\newcommand{\includecode}[3]{\lstinputlisting[caption=#1, label=#2]{#3}}

\begin{document}
\title{Linkit}
\author{Pierre Sugar\\
\texttt{pierre@sugaryourcoffee.de}}
\date{\today}
\maketitle

\begin{abstract}
\texttt{linkit} is a command line application that adds links to a web page that
is created and updated by \texttt{linkit}. Links as said can be added but also
updated and removed. It is possible to search links from the command line and
list them. \texttt{linkit} can check whether the links are still active and
list inactive links. Each link can have a category, tags and a discription. The
category is used to group links and the discription can be searched for but
tags a specifically used for searching for specific terms and topics.
\end{abstract}

\section{Project outline}
The programm runs from the command line. When provided a link \texttt{linkit}
will check whether the link exists and adds it to the web page. A link can be
anything that is accessible over an \texttt{URI}. That is web sites and files.

\begin{itemize}
  \item add a link
  \item update a link
  \item remove a link
  \item check if link is alive
  \item search link based on description and tags
  \item list all links
  \item invoke link from command line (should open web site or file)
\end{itemize}

\section{Source code management}
We organize the source in \texttt{Git} at 
\url{https://github.com/sugaryourcoffee/linkit}. We first create our project
with \texttt{Mix}, then we cd into the project directory, initialize our git
repository, do an initial commit and push the repository to \texttt{Github}.

\begin{verbatim}
$ mix new linkit
$ cd linkit
$ git init
$ git add .
$ git commit -am "initial commit"
$ git remote add origin git@github.com:sugaryourcoffee/linkit.git
$ git push -u origin master
$
\end{verbatim}

\section{Implement the Command Line Interface}
We write the test first for our command line interface to describe the
functions the command line interface actually should provide. We first 
describe the commands of \texttt{linkit}.

\begin{verbatim}
$ linkit add "http://elixir-lang.org"\
             --tag Elixir\
             --description "Home page of Elixir"
Added "http://elixir-lang.org"
$ linkit add "http://ruby-lang.org"\
             --tag Ruby\
             --description "Home of Ruby"
$ linkit update "http://elxir-lang.org"\
             --description "Home of Elixir"
Updated "http://elixir-lang.org" with description "Home of Elixir"
$ linkit delete "http://elixir-lang.org"
Deleted "http://elixir-lang.org"
$ linkit check "http://elixir-lang.com"
Check error: "http://elixir-lang.com" is not available
$ linkit list --tag "Elixir"
URL                      | Description   | Tags
-------------------------+---------------+-------
"http://elixir-lang.org" | Home of Elxir | Elixir
$ linkit call "http://elixir-lang.org"
$ linkit list
URL                      | Description   | Tags
-------------------------+---------------+-------
"http://elixir-lang.org" | Home of Elxir | Elixir
"http://ruby-lang.org"   | Home of Ruby  | Ruby   
\end{verbatim}

The test is shown in \ref{lst:cli-test} on page \pageref{lst:cli-test}.

\includecode{test/cli\_test.exs}{lst:cli-test}{files/cli_test.exs}

The implementation so far is in \ref{lst:cli} on page \pageref{lst:cli}.

\includecode{lib/linkit/cli.ex}{lst:cli}{files/cli.ex}

\end{document}
